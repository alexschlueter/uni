% Der Befehl \newcommand kann auch benutzt werden, um Variablen zu definieren:

% Nummer des Versuchs (z.B. M2):
\newcommand{\varNum}{S1}
% Name des Versuchs:
\newcommand{\varName}{Spielen, Schätzen und Statistik}
% Datum der Durchführung:
\newcommand{\varDate}{2014-10-22}
% Autoren des Protokolls:
\newcommand{\varAutor}{Alexander Schlüter, Max Mustermann}
% Nummer der eigenen Gruppe:
\newcommand{\varGruppe}{Gruppe ??}
% E-Mail-Adressen der Autoren:
\newcommand{\varEmail}{alx.schlueter@gmail.com,m\_must08@uni-muenster.de}
% E-Mail-Adressen anzeigen (true/false):
\newcommand{\varZeigeEmail}{true}
% Kopfzeile anzeigen (true/false):
\newcommand{\varZeigeKopfzeile}{true}
% Inhaltsverzeichnis anzeigen (true/false):
\newcommand{\varZeigeInhaltsverzeichnis}{true}
% Literaturverzeichnis anzeigen (true/false):
\newcommand{\varZeigeLiteraturverzeichnis}{true}


