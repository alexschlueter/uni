%        File: test.tex
%     Created: Fri Oct 10 01:00 AM 2014 C
% Last Change: Fri Oct 10 01:00 AM 2014 C
%
\documentclass[ngerman, a4paper, 12pt]{scrartcl}

\usepackage[ngerman]{babel}
\usepackage[T1]{fontenc}
\usepackage[utf8]{inputenc}
\usepackage{lmodern}
\usepackage{amsmath}
\usepackage[german]{cleveref}

\begin{document}
\paragraph{Aufgabe 1}
\begin{itemize}
    \item \textbf{fett}
    \item \textit{kursiv}
    \item \texttt{Schreibmaschine}
    \item \underline{unterstrichen}
    \item \textsc{Kapitälchen}
\end{itemize}
\begin{itemize}
    \item ``\tiny{Das ist} \small{ein Satz} \normalsize{, der diese}
        \huge{Schriftgrößen demonstriert.''}
\end{itemize}
\paragraph{Aufgabe 2}
\begin{enumerate}
    \item Physik I
    \begin{itemize}
        \item Methodik der Physik
        \item Klassische (newtonsche) Mechanik
        \item Bewegungsgleichungen
        \item Einblick in die spezielle Relativitätstheorie
    \end{itemize}
    \item Physik II
        \begin{enumerate}
            \item Kinetische Gastheorie
            \begin{enumerate}
                \item Maxwell-Boltzmann-Verteilung
                \item Ideales Gas
                \item Modelle realer Gase
            \end{enumerate}
            \item Klassische Thermodynamik
            \item Elektrostatik
            \item Magnetostatik
        \end{enumerate}
    \item Physik III
        \paragraph{Klassische Elektrodynamik}
        Maxwell-Gleichungen, Wechselstrom, Licht
        \begin{enumerate}
            \item Elektrodynamik im Vakuum
            \item Elektrodynamik in Materie
        \end{enumerate}
        \paragraph{Ausbreitung von Wellen} Wellen auf Kabeln, Lichtausbreitung im Vakuum
        \paragraph{Quantenphysik} Einblick in die Quantenmechanik
        \paragraph{Spezielle Relativitätstheorie} Schnell bewegte Bezugssysteme
\end{enumerate}
\paragraph{Aufgabe 3}
\begin{enumerate}
    \item
    \begin{equation}
        E=mc^2
    \end{equation}
    \begin{equation}
        F=-G \cdot \frac{mM}{r^2}
        \label{eq:gravi}
    \end{equation}
    \begin{equation}
        C=\frac{Q}{U}
    \end{equation}
    \begin{equation}
        U=U_0 \cdot e^{-\frac{t}{RC}}
    \end{equation}
    \begin{equation}
        U=2 \pi r
        \label{eq:umf}
    \end{equation}
    \begin{equation}
        A=\pi r^2 = \pi \frac{d^2}{4}
        \label{eq:flach}
    \end{equation}
    \begin{equation}
        r=\sqrt{\frac{(s-a)(s-b)(s-c)}{s}}
    \end{equation}
    \item
    Der Satz von Pythagoras, $a^2+b^2=c^2$, beschreibt das Verhältnis von
    Längen in einem rechtwinkligen Dreieck und ist eine sehr bekannte
    mathematische Aussage.
    \item
    \begin{align}
        (a+b)^3 &= (a+b)^2(a+b) \label{eq:ab1} \\
            &= (a^2+2ab+b^2)(a+b) \\
            &= [(a^3+2a^2b+ab^2)+(a^2b+2ab^2+b^2)] \\
            &= a^3+3a^2b+3ab^2+b^3 \label{eq:ab2}
    \end{align}
    \item
    \begin{align*}
        0 &= \frac{1}{\pi} \int_0^{2\pi} \cos(nx)\sin(mx) \, \mathrm{d}x \\
        \dot{Q} &= -\lambda \Lambda \frac{\Delta T}{L} \\
        I &= \int_A \vec{j} \cdot \mathrm{d}\vec{\Lambda} \\
        \vec{F} &= -\frac{q_1 q_2}{4 \pi
          \varepsilon_0}\frac{\vec{r_1}-\vec{r_2}}{|\vec{r_1}-\vec{r_2}|^3} \\
        \Lambda &= \frac{\tan(x)\sin(x)}{\pi^2\rho_{neu}} \cdot \rho_{alt}
          \cdot \int_1^x \exp\left( \frac{mv^2}{2\ln(x)} \right) \,
          \mathrm{d}x
    \end{align*}
\item
    \begin{align}
        \nabla\cdot \vec{D} &= \rho_f \\
        \nabla\cdot \vec{B} &= 0 \\
        \nabla\times \vec{E} &= -\frac{\partial \vec{B}}{\partial t} \\
        \nabla\times \vec{H} &= \vec{j}_f+\frac{\partial \vec{D}}{\partial t}
    \end{align}
\end{enumerate}
\paragraph{Aufgabe 4}
\begin{enumerate}
    \item Versucht, einen automatischen Verweis auf \cref{eq:gravi} zu
        erzeugen.
    \item Erzeugt den folgenden Text mit automatischen Verweisen:
        ``\cref{eq:gravi,eq:umf,eq:flach}''
    \item Erzeugt den folgenden Text mit automatischen Verweisen:
        ``\crefrange{eq:ab1}{eq:ab2}''
\end{enumerate}
\paragraph{Aufgabe 5}
\begin{enumerate}
    \item 
        \begin{tabular}{| c c || c |}
            \hline
            1 & 2 & 3 \\
            \hline \hline
            4 & 5 & 6 \\
            7 & 8 & 9 \\ \hline
        \end{tabular}
    \item
        \begin{tabular}{l | l l l l | l}
            \textbf{Haarfarbe} & \multicolumn{4}{c}{\textbf{Augenfarbe}} &
            \textbf{Summe} \\
        \end{tabular}
\end{enumerate}
\end{document}
