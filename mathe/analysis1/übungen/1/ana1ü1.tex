\documentclass[a4paper,11pt,parskip=half,fleqn]{scrartcl}

\usepackage[ngerman]{babel}         % Neue deutsche Rechtschreibung
\usepackage[T1]{fontenc}            % Bessere Schriftdarstellung
\usepackage{lmodern}                % Aktuelle Schrift


\usepackage[utf8]{inputenc}                      %Linux

%\usepackage{mathptmx}
\usepackage[intlimits, fleqn]{amsmath}     % Zusaetzliche Matheumgebungen
\usepackage{amssymb}                % Mathematische Symbole
\usepackage{amsthm}
\usepackage{siunitx}
\usepackage{graphicx}               % notwendig fuer \includegraphics
\usepackage{fancyhdr}               % Kopf- und Fusszeile
\usepackage{lastpage}               % erzeugt Referez zu der letzten Seite
\usepackage{moreverb}               % verbatimtab Umgeung
%\usepackage{enumitem}
\usepackage{framed, xcolor, color} 
\usepackage{listings}
\usepackage{caption}
\usepackage{wallpaper}
\usepackage{booktabs,tabularx,rotating}
\usepackage{nicefrac,units}
\usepackage[normalem]{ulem} %unterstreichungen 
\usepackage{paralist}
\usepackage{mathrsfs}
\usepackage{mathtools}
\usepackage{paralist}
\usepackage{pgfplots}

\DeclareSIUnit\annum{a}
\newcommand{\wrt}[1]{\mathrm{d}{#1}}
\newcommand{\diff}[1]{\frac{\mathrm{d}}{\wrt{#1}}}
\newcommand{\difff}[2]{\frac{\wrt{#1}}{\wrt{#2}}}
\newcommand*\colvec[3][]{
    \begin{pmatrix}\ifx\relax#1\relax\else#1\\\fi#2\\#3\end{pmatrix}
}

\DeclareMathAlphabet{\mathpzc}{OT1}{pzc}{m}{it}
% Seiteneinstelungen
\setlength\textwidth{165mm}           % Breite
\setlength\textheight{235mm}          % Hoehe
\setlength\headheight{41pt}           % Hoehe der Kopfzeile
\setlength\topmargin{-12mm}           % Abstand oben
\setlength\oddsidemargin{0mm}         % Linker Rand
%\setlength\parindent{0pt}             % und ohne Einrueckung
%\setlength\parskip{1.7\medskipamount} % Absaetze abgesetzt
\sloppy\pagestyle{fancy}

%Kopf- und Fusszeileeinstellungen
\renewcommand{\headrulewidth}{0.4pt} 	%obere Trennlinie
%\fancyfoot[C]{Seite:~\thepage~von~\pageref{LastPage}} %Seitennummer
\fancyfoot[C]{}
%\renewcommand{\footrulewidth}{0.4pt} 	%untere Trennlinie

\fancyhead[C]{\bf{Übung 1\\}}
\fancyhead[L]{Analysis 1\\ Alexander Schlüter}
\fancyhead[R]{\today \\ Seite \thepage}

\newcommand{\Aufgabe}[2][]{{\vskip0cm \sffamily {\Large \bfseries Aufgabe #2: #1} \par}}

\usepackage{xcolor}
\usepackage{framed}

\definecolor{pink}{cmyk}{0.38,0.87,0,0}
\definecolor{darkgreen}{cmyk}{0.86,0.2,1.0,0.07}
\definecolor{headercolor}{HTML}{2867A4}
\definecolor{shadecolor}{HTML}{D6DEE9}
\lstset{basicstyle={\ttfamily},
frame=single,
framerule=0.4pt,
breaklines=true,
commentstyle={\color{darkgreen}},
extendedchars=true,
keywordstyle={\bfseries\color{pink}},
language={C++},
numbers=left,
numberstyle={\color{gray}},
stringstyle={\color{red}},
showstringspaces=false,
tabsize=4,
numberbychapter=false}

\newtheoremstyle{note}% ⟨name⟩
{18pt}%	⟨Space above⟩ 
{9pt}%	⟨Space below⟩ 
{}%	⟨Body font⟩
{}%	⟨Indent amount⟩1 
{\bfseries}% ⟨Theorem head font⟩ 
{}%	⟨Punctuation after theorem head⟩ 
{0.5em}%	⟨Space after theorem head⟩2 
{}%	⟨Theorem head spec (can be left empty, meaning ‘normal’)⟩
\theoremstyle{note}
\newtheorem*{definition}{Definition}
\newtheorem*{bemerkung}{Bemerkung}
\newtheorem*{satz}{Satz}
\newtheorem*{lemma}{Lemma}
\newtheorem*{axiom}{Axiom}
\newtheorem*{beispiele}{Beispiele}
\newtheorem*{beispiel}{Beispiel}
\newtheorem{aufgabe}{Aufgabe}
\newtheorem*{leer}{}

%\renewcommand{\theenumi}{\roman{enumi}}
\renewcommand{\labelenumi}{(\theenumi)}

\newcommand{\Num}[1]{\text{Num}_#1}
\newcommand{\Bin}{\text{Bin}}
\newcommand{\Oct}{\text{Oct}}
\newcommand{\Hex}{\text{Hex}}
\newcommand{\id}{\text{id}}
\newcommand{\N}{\mathbb{N}}
\newcommand{\Z}{\mathbb{Z}}
\newcommand{\Q}{\mathbb{Q}}
\newcommand{\R}{\mathbb{R}}
\newcommand{\C}{\mathbb{C}}
\newcommand{\iu}{\text{i}}

\DeclarePairedDelimiter\abs{\lvert}{\rvert}%
\DeclarePairedDelimiter\norm{\lVert}{\rVert}%

% Swap the definition of \abs* and \norm*, so that \abs
% and \norm resizes the size of the brackets, and the 
% starred version does not.
\makeatletter
\let\oldabs\abs
\def\abs{\@ifstar{\oldabs}{\oldabs*}}
%
\let\oldnorm\norm
\def\norm{\@ifstar{\oldnorm}{\oldnorm*}}
\makeatother

\begin{document}
\begin{aufgabe}
  \begin{enumerate}
    \item 
      IB: $0^5-0=0=5*0$ \\
      IV: Für beliebiges, aber festes $n\in\N$ gelte die Behauptung. \\
      IS: $(n+1)^5-n-1=n^5+5n^4+10n^3+10n^2+5n-n-1\overset{IV}{=}5(n^4+2n^3+2n^2+n+\frac{n^5-n}{5})\qed$
    \item
      IB ($n=1$): $\prod\limits_{k=2}^1(1-\frac{1}{k^2})=1=\frac{1+1}{2*1}$ \\
      IV: Für beliebiges, aber festes $n\in\N$ gelte die Behauptung. \\
      IS: $\prod\limits_{k=2}^{n+1}(1-\frac{1}{k^2})\overset{IV}{=}(1-\frac{1}{(n+1)^2})\frac{n+1}{2n}=\frac{n+1}{2n}-\frac{1}{2n(n+1)}=\frac{(n+1)^2-1}{2n(n+1)}
      =\frac{n^2+2n}{2n^2+2n}=\frac{(n+1)+1}{2(n+1)}\qed$
    \item 
      IB: $\sum\limits_{k=1}^0 k^3=0=\frac{0^2(0+1)^2}{4}$ \\
      IV: Für beliebiges, aber festes $n\in\N$ gelte die Behauptung. \\
      IS: $\sum\limits_{k=1}^{n+1}k^3\overset{IV}{=}(n+1)^3+\frac{n^2(n+1)^2}{4}=\frac{n^4+6n^3+13n^2+12n+4}{4}=\frac{(n^2+2n+1)(n^2+4n+4)}{4}=\frac{(n+1)^2(n+1+1)^2}{4}\qed$
    \item Die Behauptung gilt für $n\in\{0,1,2\}$ (Beweis durch Einsetzen). Außerdem gilt sie für alle $n\geq 8$: \\
      IB: $8^3=512\leq 2^{8+1}$ \\
      IV: Für beliebiges, aber festes $n\in\N,n\geq 8$ gelte die Behauptung. \\
      IS: $(n+1)^3=n^3+3n^2+3n+1\overset{IV}{\leq} 2^{n+1}+3n^2+3n+1\overset{n\geq 8}{\leq} 2^{n+1}+3n^2+4n\overset{n\geq 8}{\leq} 2^{n+1}+4n^2\overset{n\geq 8}{\leq} 2^{n+1}+n^3\overset{IV}{\leq}2^{n+2}\qed$
  \end{enumerate}
\end{aufgabe}
\begin{aufgabe}
  \begin{enumerate}
    \item 
      IB: $\sum\limits_{k=1}^{2*1}\frac{(-1)^{k+1}}{k}=1-\frac{1}{2}=\sum\limits_{k=1+1}^{2*1}\frac{1}{k}$ \\
      IV: Für beliebiges, aber festes $n\in\N,n\geq 1$ gelte die Behauptung. \\
      IS: $\sum\limits_{k=1}^{2(n+1)}\frac{(-1)^{k+1}}{k}\overset{IV}{=}\sum\limits_{k=n+1}^{2n}\frac{1}{k}+\frac{(-1)^{2n+2}}{2n+1}+\frac{(-1)^{2n+3}}{2n+2}
      =\sum\limits_{k=n+1}^{2n}\frac{1}{k}+\frac{1}{2n+1}-\frac{1}{2n+2}=\sum\limits_{k=n+1}^{2n+1}\frac{1}{k}-\frac{1}{2n+2}
      =\sum\limits_{k=n+2}^{2n+2}\frac{1}{k}-\frac{2}{2n+2}+\frac{1}{n+1}=\sum\limits_{k=(n+1)+1}^{2(n+1)}\frac{1}{k}\qed$
    \item 
      IB: $\sum\limits_{k=1}^{1}(-1)^k k^2=-1=(-1)^1\binom{1+1}{2}$ \\
      IV: Für beliebiges, aber festes $n\in\N,n\geq 1$ gelte die Behauptung. \\
      IS: $\sum\limits_{k=1}^{n+1}(-1)^k k^2\overset{IV}{=}(-1)^n\binom{n+1}{2}+(-1)^{n+1}(n+1)^2=(-1)^{n+1}\left( (n+1)^2-\frac{n(n+1)}{2} \right)
      =(-1)^{n+1}\left( \frac{n^2+3n+2}{2} \right)=(-1)^{n+1}\left( \frac{(n+1)(n+2)}{2} \right)=(-1)^{n+1}\binom{(n+1)+1}{2}\qed$
  \end{enumerate}
\end{aufgabe}
\begin{aufgabe}
  \begin{enumerate}
    \item 
      IB: Für alle $n\in\N$ gilt $\binom{n}{0}\frac{1}{n^0}=1\leq \frac{1}{1!}$ \\
      IV: Für beliebiges, aber festes $k\in\N$ gelte die Behauptung. \\
      IS: $\binom{n}{k+1}\frac{1}{n^{k+1}}=\binom{n}{k}\frac{1}{n^k}*\frac{(n-k)}{n(k+1)}\overset{IV}{\leq} \frac{1}{k!}*\frac{n-k}{n(k+1)}=\frac{n-k}{n}\frac{1}{(k+1)!}$ \\
      $\frac{n-k}{n}\leq 1$ muss gelten, da für $k>n$ der Binomialkoeffizient nicht definiert wäre. Also gilt: \\
      $\frac{n-k}{n}\frac{1}{(k+1)!}\leq \frac{1}{k+1!}\qed$
    \item 
      IB: $2^4=16\leq24=4!$ \\
      IV: Für beliebiges, aber festes $n\in\N,n\geq 4$ gelte die Behauptung. \\
      IS: $2^{n+1}=2*2^n\overset{IV}{\leq}2*n!<(n+1)*n!$, da $2<n+1$ für $n\geq 4$ \\
      $(n+1)*n!=(n+1)!\qed$
    \item Jedes Summenglied von $(1+\frac{1}{n})^n=\sum\limits_{k=0}^n\binom{n}{k}\frac{1}{n^k}$ entspricht der linken Seite aus 3a), welches jeweils einem
      Summenglied von $\sum\limits_{k=0}^n \frac{1}{k!}$ zugeordnet werden kann (vgl. 3a) rechte Seite). Die Summanden auf der linken Seite sind jeweils
      kleiner gleich den Zugeordneten auf der rechten Seite (Beweis geliefert in 3a)) \\
      Außerdem ist $\sum\limits_{k=0}^\infty \frac{1}{k!}=e=2.718\dots<3\qed$
    \item 
      IB: $\left(\frac{1}{3}\right)^1=\frac{1}{3}\leq \frac{1}{3}*1!$ \\
      IV: Für beliebiges, aber festes $n\in\N,n\geq 1$ gelte die Behauptung. \\
      IS: Nach c) gilt $( \frac{n+1}{n})^n=(1+\frac{1}{n})^n\overset{c)}{<}3$
      \begin{equation*}\begin{split}
	\overset{IV}{\implies}\left( \frac{n}{3}\right)^n*\left( \frac{n+1}{n}\right)^n<\frac{1}{3}n!*3\iff \frac{n^n(n+1)^n}{3^n*n^n}= \\
	\left( \frac{n+1}{3}\right)^n<n!\iff\left(\frac{n+1}{3}\right)^{n+1}<n!*\frac{n+1}{3}=\frac{1}{3}(n+1)!\qed
      \end{split}\end{equation*}
  \end{enumerate}
\end{aufgabe}
\begin{aufgabe}
  In der Vorlesung wurde gezeigt, dass die Anzahl aller $k$-elementigen Teilmengen einer $n$-elementigen Menge $\binom{n}{k}$ beträgt. Die Summe dieser
  Teilmengen $\sum_{k=0}^n \binom{n}{k}$ ist die Anzahl aller Teilmengen, also gesuchte Ordnung der Potenzmenge. Es bleibt zu zeigen, dass
  diese Summe $2^n$ ergibt:
  \begin{equation*}\begin{aligned}
    &\text{IB:\quad}\sum_{k=0}^0\binom{0}{k}=1=2^0 \\
    &\text{IV:\quad Für beliebiges, aber festes $n\in\N$ gelte die Behauptung.} \\
    &\text{IS:\quad}\sum_{k=0}^{n+1}\binom{n+1}{k}=2+\sum_{k=1}^{n}\binom{n+1}{k}\overset{\text{Vorlesung}}{=}2+\sum_{k=1}^{n}\binom{n}{k-1}+\sum_{k=1}^{n}\binom{n}{k} \\
    &=1+\sum_{k=1}^{n}\binom{n}{k-1}+\sum_{k=0}^{n}\binom{n}{k}\overset{IV}{=}1+\sum_{k=1}^{n}\binom{n}{k-1}+2^n=1+\sum_{k=0}^{n-1}\binom{n}{k}
    +2^n \\
    &=\sum_{k=0}^n\binom{n}{k}+2^n\overset{IV}{=}2^n+2^n=2^{n+1}\qed \\
  \end{aligned}\end{equation*}
\end{aufgabe}
\end{document}
