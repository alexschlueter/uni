\documentclass[a4paper,11pt,parskip=half,fleqn]{scrartcl}

\usepackage[ngerman]{babel}         % Neue deutsche Rechtschreibung
\usepackage[T1]{fontenc}            % Bessere Schriftdarstellung
\usepackage{lmodern}                % Aktuelle Schrift


\usepackage[utf8]{inputenc}                      %Linux

%\usepackage{mathptmx}
\usepackage[intlimits, fleqn]{amsmath}     % Zusaetzliche Matheumgebungen
\usepackage{amssymb}                % Mathematische Symbole
\usepackage{amsthm}
\usepackage{siunitx}
\usepackage{graphicx}               % notwendig fuer \includegraphics
\usepackage{fancyhdr}               % Kopf- und Fusszeile
\usepackage{lastpage}               % erzeugt Referez zu der letzten Seite
\usepackage{moreverb}               % verbatimtab Umgeung
%\usepackage{enumitem}
\usepackage{framed, xcolor, color} 
\usepackage{listings}
\usepackage{caption}
\usepackage{wallpaper}
\usepackage{booktabs,tabularx,rotating}
\usepackage{nicefrac,units}
\usepackage[normalem]{ulem} %unterstreichungen 
\usepackage{paralist}
\usepackage{mathrsfs}
\usepackage{mathtools}
\usepackage{paralist}
\usepackage{pgfplots}

\DeclareSIUnit\annum{a}
\newcommand{\wrt}[1]{\mathrm{d}{#1}}
\newcommand{\diff}[1]{\frac{\mathrm{d}}{\wrt{#1}}}
\newcommand{\difff}[2]{\frac{\wrt{#1}}{\wrt{#2}}}
\newcommand*\colvec[3][]{
    \begin{pmatrix}\ifx\relax#1\relax\else#1\\\fi#2\\#3\end{pmatrix}
}

\DeclareMathAlphabet{\mathpzc}{OT1}{pzc}{m}{it}
% Seiteneinstelungen
\setlength\textwidth{165mm}           % Breite
\setlength\textheight{235mm}          % Hoehe
\setlength\headheight{41pt}           % Hoehe der Kopfzeile
\setlength\topmargin{-12mm}           % Abstand oben
\setlength\oddsidemargin{0mm}         % Linker Rand
%\setlength\parindent{0pt}             % und ohne Einrueckung
%\setlength\parskip{1.7\medskipamount} % Absaetze abgesetzt
\sloppy\pagestyle{fancy}

%Kopf- und Fusszeileeinstellungen
\renewcommand{\headrulewidth}{0.4pt} 	%obere Trennlinie
%\fancyfoot[C]{Seite:~\thepage~von~\pageref{LastPage}} %Seitennummer
\fancyfoot[C]{}
%\renewcommand{\footrulewidth}{0.4pt} 	%untere Trennlinie

\fancyhead[C]{\bf{Übung 1\\}}
\fancyhead[L]{Lineare Algebra 1\\ Alexander Schlüter}
\fancyhead[R]{\today \\ Seite \thepage}

\newcommand{\Aufgabe}[2][]{{\vskip0cm \sffamily {\Large \bfseries Aufgabe #2: #1} \par}}

\usepackage{xcolor}
\usepackage{framed}

\definecolor{pink}{cmyk}{0.38,0.87,0,0}
\definecolor{darkgreen}{cmyk}{0.86,0.2,1.0,0.07}
\definecolor{headercolor}{HTML}{2867A4}
\definecolor{shadecolor}{HTML}{D6DEE9}
\lstset{basicstyle={\ttfamily},
frame=single,
framerule=0.4pt,
breaklines=true,
commentstyle={\color{darkgreen}},
extendedchars=true,
keywordstyle={\bfseries\color{pink}},
language={C++},
numbers=left,
numberstyle={\color{gray}},
stringstyle={\color{red}},
showstringspaces=false,
tabsize=4,
numberbychapter=false}

\newtheoremstyle{note}% ⟨name⟩
{18pt}%	⟨Space above⟩ 
{9pt}%	⟨Space below⟩ 
{}%	⟨Body font⟩
{}%	⟨Indent amount⟩1 
{\bfseries}% ⟨Theorem head font⟩ 
{}%	⟨Punctuation after theorem head⟩ 
{0.5em}%	⟨Space after theorem head⟩2 
{}%	⟨Theorem head spec (can be left empty, meaning ‘normal’)⟩
\theoremstyle{note}
\newtheorem*{definition}{Definition}
\newtheorem*{bemerkung}{Bemerkung}
\newtheorem*{satz}{Satz}
\newtheorem*{lemma}{Lemma}
\newtheorem*{axiom}{Axiom}
\newtheorem*{beispiele}{Beispiele}
\newtheorem*{beispiel}{Beispiel}
\newtheorem{aufgabe}{Aufgabe}
\newtheorem*{leer}{}

%\renewcommand{\theenumi}{\roman{enumi}}
\renewcommand{\labelenumi}{(\theenumi)}

\newcommand{\Num}[1]{\text{Num}_#1}
\newcommand{\Bin}{\text{Bin}}
\newcommand{\Oct}{\text{Oct}}
\newcommand{\Hex}{\text{Hex}}
\newcommand{\id}{\text{id}}
\newcommand{\N}{\mathbb{N}}
\newcommand{\Z}{\mathbb{Z}}
\newcommand{\Q}{\mathbb{Q}}
\newcommand{\R}{\mathbb{R}}
\newcommand{\C}{\mathbb{C}}
\newcommand{\iu}{\text{i}}

\DeclarePairedDelimiter\abs{\lvert}{\rvert}%
\DeclarePairedDelimiter\norm{\lVert}{\rVert}%

% Swap the definition of \abs* and \norm*, so that \abs
% and \norm resizes the size of the brackets, and the 
% starred version does not.
\makeatletter
\let\oldabs\abs
\def\abs{\@ifstar{\oldabs}{\oldabs*}}
%
\let\oldnorm\norm
\def\norm{\@ifstar{\oldnorm}{\oldnorm*}}
\makeatother

\begin{document}
\begin{aufgabe}
  \begin{enumerate}
    \item $x\in A\cap (B\cup C)\iff x\in A\land (x\in B\lor x\in C)\iff (x\in A\land x\in B)\lor (x\in A\land x\in C)\iff x\in (A\cap B)\cup (A\cap C)$
    \item $x\in A\cup (B\cap C)\iff x\in A\lor (x\in B\land x\in C)\iff (x\in A\lor x\in B)\land (x\in A\lor x\in C)\iff x\in (A\cup B)\cap (A\cup C)$
    \item $x\in A-(B\cup C)\iff x\in A\land \neg (x\in B\lor x\in C)\iff x\in A\land (x\not\in B\land x\not\in C)\iff
      (x\in A\land x\not\in B)\land (x\in A\land x\not\in C)\iff x\in (A-B)\cup (A-C)$
    \item $x\in A-(B\cap C)\iff x\in A\land \neg (x\in B\land x\in C)\iff x\in A\land (x\not\in B\lor x\not\in C)\iff
      (x\in A\land x\not\in B)\lor (x\in A\land x\not\in C)\iff x\in (A-B)\cup (A-C)$
  \end{enumerate}
\end{aufgabe}
\begin{aufgabe}
  Die Äquivalenz aller Aussagen mit $x\in B\lor x\not\in A$ wird gezeigt:
  \begin{enumerate}
    \item
      \begin{align*}
	&A\subseteq B \\
	\iff& (x\in A\implies x\in B) \\
	\iff& x\in B\lor x\not\in A
      \end{align*}
    \item
      \begin{align*}
	&A\cap B=A \\
	\iff& (x\in A\land x\in B\iff x\in A) \\
	\iff& (x\in A\land x\in B\land x\in A)\lor ((x\not\in A\lor x\not\in B)\land x\not\in A) \\
	\iff& (x\in A\land x\in B)\lor ((x\not\in A\lor x\not\in B)\land x\not\in A) \\
	\iff& (x\in A\land x\in B)\lor x\not\in A \\
	\iff& (x\in A\lor x\not\in A)\land (x\in B\lor x\not\in A) \\
	\iff& x\in B\lor x\not\in A
      \end{align*}
    \item
      \begin{align*}
	&A\cup B=B \\
	\iff& (x\in A\lor x\in B\iff x\in B) \\
	\iff& ((x\in A\lor x\in B)\land x\in B)\lor ((x\not\in A\land x\not\in B)\land x\not\in B) \\
	\iff& x\in B\lor (x\not\in A\land x\not\in B) \\
	\iff& (x\in B\lor x\not\in A)\land (x\in B\lor x\not\in B) \\
	\iff& x\in B\lor x\not\in A \\
      \end{align*}
    \item
      \begin{align*}
	&A-B=\emptyset \\
	\iff& (x\in A\land x\not\in B\iff x\in\emptyset) \\
	\iff& (x\in A\land x\not\in B\land x\in\emptyset)\lor ((x\not\in A\lor x\in B)\land x\not\in\emptyset) \\
	\iff& x\in B\lor x\not\in A
      \end{align*}
  \end{enumerate}
\end{aufgabe}
\begin{aufgabe}
  \begin{minipage}[t]{0.25\textwidth}
    $p$: Anzahl Pferde
  \end{minipage}
  \begin{minipage}[t]{0.25\textwidth}
    $k$: Anzahl Kühe
  \end{minipage}
  \begin{minipage}[t]{0.25\textwidth}
    $h$: Anzahl Hühner
  \end{minipage} \\
  Nach den Bedingungen gilt $k(k+p)=h+120$. \\
  \paragraph*{Fall 1:}
  Beide Seiten der Gleichung sind gerade. \\
  $h+120$ ist gerade, wenn $h$ gerade ist. Die einzige gerade Primzahl ist $2$, es gilt also $h=2$. \\
  Es folgt $k(k+p)=122=2*61$. $k\neq 2$, da verschiedene Primzahlen gesucht werden und $h=2$ angenommen wird. Es muss also gelten $k+p=2$, was nicht durch zwei Primzahlen zu erfüllen ist $\implies$ Widerspruch.
  \paragraph*{Fall 2:}
  Es müssen also beide Seiten der Gleichung ungerade sein. \\
  $k(k+p)$ ist ungerade, wenn beide Faktoren ungerade sind. $k+p$ ist ungerade, wenn genau einer der Summanden ungerade ist. Da $k$ bereits als Faktor auftritt und zwingend
  ungerade sein muss, folgt $p$ ist gerade und damit $p=2$.
  \begin{align*}
    &k(k+2)=k^2+2k=h+120\iff k^2+2k-120=h=(k+12)(k-10)
  \end{align*}
  Einer der Teiler der Primzahl $h$ muss $1$ sein, und $k+12$ ist es nicht, da sonst $k=-11$ wäre. Also bleibt $1=k-10\iff k=11$. \\
  Durch Einsetzen folgt $h=23$. Der Bauer hat 2 Pferde, 11 Kühe und 23 Hühner.
\end{aufgabe}
\begin{aufgabe}
  Es gilt die Formel $F_n^2=F_{n-1}F_{n+1}+(-1)^{n-1}$\quad: \\
  IB: $F_2^2=1*1=1=1*2-1=F_1F_3+(-1)^1$ \\
  IV: Die Behauptung gelte für ein beliebiges, aber festes $n\in\N,n\geq 2$. \\
  IS:
  \begin{align*}
    &F_{n+1}^2=F_{n+1}(F_n+F_{n-1})=F_nF_{n+1}+F_{n-1}F_{n+1}\overset{IV}{=}F_nF_{n+1}+F_n^2+(-1)^n \\
    &=F_n(F_{n+1}+F_n)+(-1)^n=F_{(n+1)-1}F_{(n+1)+1}+(-1)^{(n+1)-1}\qed
  \end{align*}
\end{aufgabe}
\end{document}
