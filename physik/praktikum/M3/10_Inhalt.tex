\section{Einführung}
Wird ein starrer Körper reversibel verformt, so nennt man dies eine \textbf{elastische} Verformung.
Dabei treten im Körper Spannungen auf, die der Verformungskraft entgegenwirken (3. Newtonsches Gesetz).

Das \textbf{Hookesche Gesetz} nimmt einen proportionalen Zusammenhang zwischen in der Stärke der Verformung und der im Körper wirkenden Spannung an.
Im Falle einer Zugkraft hat dieses die Form
\begin{equation}
  \sigma = E\cdot \varepsilon,
  \label{eq:hookzug}
\end{equation}
wobei $\sigma$ elastische Zugspannung, $E$ Elastizitätsmodul und $\varepsilon=\frac{\Delta L}{L}$ relative Längenänderung heißen.
Bei einer Scherkraft lautet es
\begin{equation}
  \tau = G\alpha,
  \label{eq:hookscher}
\end{equation}
wobei $\tau$ die Schubspannung, $G$ das Schubmodul und $\alpha$ der Scherwinkel sind. \\

Neben einer Längenänderung kommt es auch zu einer Verkürung der Querabmessung $R$ (z.B. Radius). Die \textbf{Poissonzahl / Querkontraktionszahl} gibt das Verhältnis der relativen Änderungen an:
\begin{equation}
  \mu=-\frac{\Delta R/R}{\Delta L/L}
  \label{eq:poisson}
\end{equation}
Wird ein allseitiger Druck $p$ auf einen Körper ausgeübt, so heisst der Proportionalitätsfaktor zur relativen Volumenänderung \textbf{Kompressionsmodul} ($K$):
\begin{equation}
  p=-K \frac{\Delta V}{V}
  \label{eq:kompression}
\end{equation}
Die Größen $E, \mu, G, K$ sind also Materialkonstanten, die das Verhalten eines Körpers unter äußeren Verformungskräften beschreiben, solange diese elastisch bleiben. \\

Wird ein Körper an einem Ende festgespannt und am anderen, freien Ende eine Kraft angelegt, so spricht man von einer \textbf{Elastischen Biegung}.
Dem äußeren Drehmoment
\begin{equation}
  M_a=(L-z)F
  \label{eq:drehmom}
\end{equation}
($L$ Länge des Körpers, $z$ Abstand von festem Ende) wird ein inneres Drehmoment entgegengesetzt, welches sich berechnen lässt als
\begin{equation}
  M_i=\frac{E}{R}\cdot I_q
  \label{eq:in_drehmom}
\end{equation}
($E$ Elastizitätsmodul, $R$ lokaler Krümmungsradius). Dabei ist $I_q$ das Flächenträgheitsmoment, welches vom Querschnitt des Körpers abhängt:
\begin{align}
  I_q &= \int y^2 \, \mathrm{d}A \\
  I_{\text{Kreis}}&=\frac{\pi d^4}{64} \qquad \text{$d$ Durchmesser} \\
  I_{\text{Rechteck}}&=\frac{ab^3}{12} \qquad \text{$a$ senkrechte Kantenlänge, $b$ parallele}
  \label{eq:flaechentraegheit}
\end{align}
Die maximale Durchbiegung am freien Ende des Stabes ist
\begin{equation}
  h_{\text{max}}=\frac{F}{E I_q}\frac{L^3}{3}
  \label{eq:maxbiegung}
\end{equation}
\\
Eine \textbf{Elastische Torsion} ist eine Verdrillung eines zylindrischen Stabes (Radius $R$, Länge $L$) um eine Torsionswinkel $\varphi$. Es können wieder äußeres und inneres Drehmoment gleichgesetzt werden:
\begin{equation}
  M_a = RF=\frac{\pi G R^4}{2L}\varphi=M_i
  \label{eq:torsion_drehmom}
\end{equation}
Hängt man einen Körper mit Trägheitsmoment $J$ an den Stab und verdreht diesen um $\varphi$, so kommt es zu einer harmonischen Schwingung:
\begin{equation}
  0=J\ddot{\varphi}+D^* \varphi
  \label{eq:schwingung}
\end{equation}
$D^*$ heisst Direktionsmoment. Aus der Periodendauer $T$ kann das Schubmodul errechnet werden:
\begin{equation}
  G= \frac{8\pi LJ}{R^4T^2}
  \label{eq:schubmod}
\end{equation}

Das zur Rechnung benötigte Trägheitsmoment lautet für eine Scheibe mit Masse $m$ und Radius $R$:
\begin{equation}
  J_{\text{Scheibe}}=\frac{1}{2}mR^2
  \label{eq:j_scheibe}
\end{equation}
Um dieses für eine Hantel zu bestimmen, hilft der \textbf{Steinerschen Satz}. Hat ein Körper mit Masse $m$ ein Trägheitsmoment $J_S$ bez. eines Punktes $S$, der im Abstand $a$ zu einem zweiten Punkt $A$ liegt, so lautet das Trägheitsmoment bez. $A$:
\begin{equation}
  J_A=J_S+a^2m
  \label{eq:steiner}
\end{equation}
Es gilt nun für eine Hantel:
\begin{align}
  J_{\text{Achse}}&=m_1\left( \frac{1}{12}l_1^2+\frac{1}{4}r_1^2 \right) \\
  J_{\text{Scheiben}}&=m_2\left(\frac{1}{12}l_2^2+\frac{1}{4}(r_2^2+r_1^2)\right) \\
  J_{\text{Hantel}}&=J_{\text{Achse}}+2\cdot J_{\text{Scheiben}}+2m_2a^2
  \label{eq:hantel}
\end{align}
Wird diese als Torsionspendel genutzt, ergibt sich für die Schwigung folgende Beziehung zwischen Periodendauer und Abstand der Scheiben:
\begin{equation}
  T^2=\frac{4\pi^2}{D^*}(J_{\text{Achse}}+2\cdot J_{\text{Scheiben}}+2m_2a^2)
  \label{eq:hanteldauer}
\end{equation}
\section{Versuch: Elastische Biegung}
Es wurden 4 Metallstäbe zur Verfügung gestellt:
\begin{table}[h]
  \centering
  \begin{tabular}{l | c | c | r}
    \# & Aussehen & Form & gefühltes Gewicht \\ \hline
    1 & Gold & rechteckig & - \\
    2 & Gold & zylindrisch & mittel \\
    3 & Silber, hell & zylindrisch & leicht \\
    4 & Silber, dunkel & zylindrisch & schwer
  \end{tabular}
  \caption{Erste Beobachtungen zu den Metallstäben}
  \label{tab:metall_beob}
\end{table}

Mit einer Mikrometerschraube wurde die Dicke der Stäbe an 5 verschiedenen Stellen je 3 mal gemessen (Messwerte siehe Laborbuch):
\begin{table}[H]
  \centering
  \begin{tabular}{l | c | c | r}
    \# & Mittelwert [mm] & Standardabweichung [mm] & Fehler [$\pm$mm] \\ \hline
    1 flachkant & \num{1.96} & \num{0.007} & \num{0.002} \\
    1 hochkant & \num{4.96} & \num{0.008} & \num{0.002} \\
    2 & \num{2.93} & \num{0.012} & \num{0.003} \\
    3 & \num{2.94} & \num{0.006} & \num{0.002} \\
    4 & \num{2.95} & \num{0.012} & \num{0.003}
  \end{tabular}
  \caption{Dicke der Stäbe}
  \label{tab:stabdicke}
\end{table}

Die Länge der Stäbe wurde mit einem Maßband gemessen (Fehler jeweils $\pm \SI{0.1}{\cm}$):
\begin{table}[H]
  \centering
  \begin{tabular}{l | c | c | r}
    \# & Länge [cm] \\ \hline
    1 & \SI{29.4}{\cm} \\
    2 & \SI{29.5}{\cm} \\
    3 & \SI{29.8}{\cm} \\
    4 & \SI{29.2}{\cm}
  \end{tabular}
  \caption{Länge der Stäbe}
  \label{tab:stablänge}
\end{table}

Weiterhin standen 5 verschiedene Gewichte (\SI{5}{g}, \SI{10}{g}, \SI{20}{g}, \SI{50}{g}, \SI{100}{g}) zur Verfügung. Die Stäbe wurden an einem Ende horizontal zum Tisch fest eingespannt, wobei das freie Ende sich vor einem Spiegel mit vertikaler Längenskala befand. 

Pro Gewicht wurde nun die Durchbiegung des Stabes gemessen: Zuerst wurde vor jeder Messung der Nullpunkt des Spiegels justiert, sodass er sich mittig hinter dem freien Stabende befand. Dann wurde das Gewicht in eine kleine Schaukel gelegt und diese am freien Ende angebracht. Eventuelle Schwingungen des Stabendes wurden abgewartet und schließlich die vertikale Auslenkung des Stabendes relativ zum Nullpunkt abgelesen. Dabei wurde der Parallaxenfehler gering gehalten, indem beim Ablesen die Spiegelung des Stabes mit dem Stab selbst ausgerichtet wurde.

Der rechteckige Stab wurde sowohl hochkant als auch flachkant eingespannt und gemessen. \\

Die Erwartung ist, dass die Stäbe sich bei höherem angehängten Gewicht stärker biegen. Laut Theorie (\cref{eq:maxbiegung}) sollte der Zusammenhang linear sein (denn $F=mg$). Möglicherweise sind anfangs als schwerer empfundenen Stäbe schlechter zu biegen als die leichteren, weil ein Zusammenhang zwischen innerer Stabilität und Dichte bestehen könnte. Außerdem wird bei flachkantigem Einspannen des 1. Stabes eine stärkere Durchbiegung als bei hochkantigen erwartet.

\begin{table}[H]
  \centering
  \begin{tabular}{l | c | c | c | c | r}
    \# & \SI{5}{g} [mm] & \SI{10}{g} [mm] & \SI{20}{g} [mm] & \SI{50}{g} [mm] & \SI{100}{g} [mm] \\ \hline
    1 flachkant & 1 & 3 & 6 & 14 & 27 \\
    1 hochkant & 0 & 1 & 2 & 3 & 5 \\
    2 & 1 & 2 & 4 & 11 & 22 \\
    3 & 2 & 4 & 7 & 16 & 31 \\
    4 & 1 & 2 & 3 & 5 & 10
  \end{tabular}
  \caption{Messergebnis zur Durchbiegung}
  \label{tab:durchbiegung}
\end{table}
\begin{figure}[H]
  \centering
  \begin{tikzpicture}
    \begin{axis}[
      width=15 cm,
      height=10 cm,
      xmin=0, xmax=100,
      ymin=0, ymax=35,
      xlabel={Gewicht [g]},
      ylabel={Durchbiegung [mm]},
      legend entries={Stab 1 flachkant, Stab 1 hochkant, Stab 2, Stab 3, Stab 4}
    ]
      \addplot[mark=square, only marks, error bars/.cd, y dir=both, y fixed=0.5] table {stab1flach.txt};
      \addplot[mark=*, only marks, error bars/.cd, y dir=both, y fixed=0.5] table {stab1hoch.txt};
      \addplot[mark=x, only marks, error bars/.cd, y dir=both, y fixed=0.5] table {stab2.txt};
      \addplot[mark=+, only marks, error bars/.cd, y dir=both, y fixed=0.5] table {stab3.txt};
      \addplot[mark=otimes, only marks, error bars/.cd, y dir=both, y fixed=0.5] table {stab4.txt};
    \end{axis}
  \end{tikzpicture}
  \caption{Messergebnis zur Durchbiegung}
  \label{fig:durchbiegung}
\end{figure}

Wie erwartet scheint die Durchbiegung linear mit dem angehängten Gewicht zu steigen. Stab 1 hat sich flachkant stärker durchgebogen als hochkant. Der leichteste Stab 3 hat sich stärker als der mittelschwere (2) und dieser wiederum stärker als der schwerste Stab 4 verbogen. \\

Mit \emph{gnuplot} werden nach dem \emph{least-squares}-Verfahren die Werte der einzelnen Stäbe gegen die Funktion $f(x)=a\cdot x$ gefittet. Ausgabe:
\begin{table}[H]
  \centering
  \begin{tabular}{l | c | c}
    \# & a [\si{mm/g}] & Varianz der Residuen \\ \hline
    1 flachkant & \num{0.3} & \num{0.2} \\
    1 hochkant & \num{0.1} & \num{0.3} \\
    2 & \num{0.2} & \num{0.1} \\
    3 & \num{0.3} & \num{.4} \\
    4 & \num{.1} & \num{.5}
  \end{tabular}
  \caption{Linearer Fit der Durchbiegung}
  \label{tab:durchbiegungsfit}
\end{table}
\cref{eq:maxbiegung} wird umgestellt zu 
\begin{align}
  h_{\text{max}}&=a\cdot m \implies a = \frac{gL^3}{3EI_q} \\
  \implies E&=\frac{mg}{aI_q}\frac{L^3}{3}
  \label{eq:durchbieg_elasti}
\end{align}
und aus der Steigung das Elastizitätsmodul berechnet:
\begin{table}[H]
  \centering
  \begin{tabular}{l | c}
    \# & Elastizitätsmodul [\si{kN/mm^2}] \\ \hline
    1 flachkant & \num{89}
  \end{tabular}
  \caption{Elastizitätsmodul berechnet aus Steigung}
  \label{tab:elastimodul}
\end{table}
\section{Abschließende Diskussion}

In der Experimentalphysik ist jeder Messwert mit einem Messfehler behaftet, der sich aus den Komponenten grober, systematischer und statistischer Fehler zusammensetzt. Um dem tatsächlichen Wert möglichst nahe zu kommen, sollten diese minimiert werden:

\begin{enumerate}
  \item \emph{Grobe} Messfehler sollten ganz vermieden werden, z.B. durch gründlichere Versuchsdurchführung und -protokollierung
  \item \emph{Systematische} Messfehler sollten klein gehalten oder nachträglich korrigiert werden, z.B. indem Messgeräte besser geeicht oder Abweichungen vor der eigentlichen Messreihe festgestellt werden
  \item \emph{Statistische} Messfehler können durch eine größere Anzahl von Messungen vermindert werden. Außerdem ist der Fehler einer zusammengesetzten Größe geringer, wenn möglichst wenig fehlerbehaftete Größen in die Berechung einfließen ($\rightarrow$ Fehlerfortpflanzung)
\end{enumerate}

Die Wahl der Messmethode kann dabei entscheidend die Größe des Fehlers beeinflussen, sodass zur Messung derselben physikalischen Größe eine Methode völlig unbrauchbar sein kann, während eine andere einen sehr genauen Wert liefert. 
Nach einer Messung muss der Fehler immer mit dem Mittelwert und der Anzahl der Messungen zusammen angegeben werden.

Sollen aus einer Messung physikalische Schlussfolgerungen gezogen werden, geschieht dies immer mit einer bestimmten Signifikanz. Diese ist groß, wenn die Wahrscheinlichkeit, ein solches Ergebnis durch Zufall zu erlangen, gering ist.

\nocite{anleitung-ws2014}
