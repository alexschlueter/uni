\section{Einführung}
Wird ein Festkörper reversibel verformt, so nennt man dies eine \textbf{elastische} Verformung.
Dabei treten im Körper Spannungen auf, die der Verformungskraft entgegenwirken (3. Newtonsches Gesetz).

Das \textbf{Hookesche Gesetz} nimmt einen proportionalen Zusammenhang zwischen in der Stärke der Verformung und der im Körper wirkenden Spannung an.
Im Falle einer Zugkraft hat dieses die Form
\begin{equation}
  \sigma = E\cdot \varepsilon,
  \label{eq:hookzug}
\end{equation}
wobei $\sigma$ elastische Zugspannung, $E$ Elastizitätsmodul und $\varepsilon=\frac{\Delta L}{L}$ relative Längenänderung heißen.
Bei einer Scherkraft lautet es
\begin{equation}
  \tau = G\alpha,
  \label{eq:hookscher}
\end{equation}
wobei $\tau$ die Schubspannung, $G$ das Schubmodul und $\alpha$ der Scherwinkel sind. \\

Neben einer Längenänderung kommt es auch zu einer Verkürung der Querabmessung $R$ (z.B. Radius). Die \textbf{Poissonzahl / Querkontraktionszahl} gibt das Verhältnis der relativen Änderungen an:
\begin{equation}
  \mu=-\frac{\Delta R/R}{\Delta L/L}
  \label{eq:poisson}
\end{equation}
Wird ein allseitiger Druck $p$ auf einen Körper ausgeübt, so heisst der Proportionalitätsfaktor zur relativen Volumenänderung \textbf{Kompressionsmodul} ($K$):
\begin{equation}
  p=-K \frac{\Delta V}{V}
  \label{eq:kompression}
\end{equation}
Die Größen $E, \mu, G, K$ sind also Materialkonstanten, die das Verhalten eines Körpers unter äußeren Verformungskräften beschreiben, solange diese elastisch bleiben. \\

Wird ein Körper an einem Ende festgespannt und am anderen, freien Ende eine Kraft angelegt, so spricht man von einer \textbf{Elastischen Biegung}.
Dem äußeren Drehmoment
\begin{equation}
  M_a=(L-z)F
  \label{eq:drehmom}
\end{equation}
($L$ Länge des Körpers, $z$ Abstand von festem Ende) wird ein inneres Drehmoment entgegengesetzt, welches sich berechnen lässt als
\begin{equation}
  M_i=\frac{E}{R}\cdot I_q
  \label{eq:in_drehmom}
\end{equation}
($E$ Elastizitätsmodul, $R$ lokaler Krümmungsradius). Dabei ist $I_q$ das Flächenträgheitsmoment, welches vom Querschnitt des Körpers abhängt:
\begin{align}
  I_q &= \int y^2 \, \mathrm{d}A \\
  I_{\text{Kreis}}&=\frac{\pi d^4}{64} \qquad \text{$d$ Durchmesser} \\
  I_{\text{Rechteck}}&=\frac{ab^3}{12} \qquad \text{$a$ senkrechte Kantenlänge, $b$ parallele}
  \label{eq:flaechentraegheit}
\end{align}
Die maximale Durchbiegung am freien Ende des Stabes ist
\begin{equation}
  h_{\text{max}}=\frac{F}{E I_q}\frac{L^3}{3}
  \label{eq:maxbiegung}
\end{equation}
\\
Eine \textbf{Elastische Torsion} ist eine Verdrillung eines zylindrischen Stabes (Radius $R$, Länge $L$) um eine Torsionswinkel $\varphi$. Es können wieder äußeres und inneres Drehmoment gleichgesetzt werden:
\begin{equation}
  M_a = RF=\frac{\pi G R^4}{2L}\varphi=M_i
  \label{eq:torsion_drehmom}
\end{equation}
Hängt man einen Körper mit Trägheitsmoment $J$ an den Stab und verdreht diesen um $\varphi$, so kommt es zu einer harmonischen Schwingung:
\begin{equation}
  0=J\ddot{\varphi}+D^* \varphi
  \label{eq:schwingung}
\end{equation}
$D^*$ heisst Direktionsmoment. Aus der Periodendauer $T$ kann das Schubmodul errechnet werden:
\begin{equation}
  G= \frac{8\pi LJ}{R^4T^2}
  \label{eq:schubmod}
\end{equation}

Das zur Rechnung benötigte Trägheitsmoment lautet für eine Scheibe mit Masse $m$ und Radius $R$:
\begin{equation}
  J_{\text{Scheibe}}=\frac{1}{2}mR^2
  \label{eq:j_scheibe}
\end{equation}
Um dieses für eine Hantel zu bestimmen, hilft der \textbf{Steinerschen Satz}. Hat ein Körper mit Masse $m$ ein Trägheitsmoment $J_S$ bez. eines Punktes $S$, der im Abstand $a$ zu einem zweiten Punkt $A$ liegt, so lautet das Trägheitsmoment bez. $A$:
\begin{equation}
  J_A=J_S+a^2m
  \label{eq:steiner}
\end{equation}
Es gilt nun für eine Hantel:
\begin{align}
  J_{\text{Achse}}&=m_1\left( \frac{1}{12}l_1^2+\frac{1}{4}r_1^2 \right) \\
  J_{\text{Scheiben}}&=m_2\left(\frac{1}{12}l_2^2+\frac{1}{4}(r_2^2+r_1^2)\right) \\
  J_{\text{Hantel}}&=J_{\text{Achse}}+2\cdot J_{\text{Scheiben}}+2m_2a^2
  \label{eq:hantel}
\end{align}
Wird diese als Torsionspendel genutzt, ergibt sich für die Schwigung folgende Beziehung zwischen Periodendauer und Abstand der Scheiben:
\begin{equation}
  T^2=\frac{4\pi^2}{D^*}(J_{\text{Achse}}+2\cdot J_{\text{Scheiben}}+2m_2a^2)
  \label{eq:hanteldauer}
\end{equation}
\section{Versuch: Elastische Biegung}
Es wurden 4 Metallstäbe zur Verfügung gestellt:
\begin{table}[h]
  \centering
  \begin{tabular}{l | c | c | r}
    \# & Aussehen & Form & gefühltes Gewicht \\ \hline
    1 & Gold & rechteckig & - \\
    2 & Gold & zylindrisch & mittel \\
    3 & Silber, hell & zylindrisch & leicht \\
    4 & Silber, dunkel & zylindrisch & schwer
  \end{tabular}
  \caption{Erste Beobachtungen zu den Metallstäben}
  \label{tab:metall_beob}
\end{table}

Mit einer Mikrometerschraube wurde die Dicke der Stäbe an 5 verschiedenen Stellen je 3 mal gemessen (Messwerte siehe Laborbuch):
\begin{table}[H]
  \centering
  \begin{tabular}{l | c | c | r}
    \# & Mittelwert [mm] & Standardabweichung [mm] & Fehler [$\pm$mm] \\ \hline
    1 flachkant & \num{1.96} & \num{0.007} & \num{0.002} \\
    1 hochkant & \num{4.96} & \num{0.008} & \num{0.002} \\
    2 & \num{2.93} & \num{0.012} & \num{0.003} \\
    3 & \num{2.94} & \num{0.006} & \num{0.002} \\
    4 & \num{2.95} & \num{0.012} & \num{0.003}
  \end{tabular}
  \caption{Dicke der Stäbe}
  \label{tab:stabdicke}
\end{table}

Die Länge der Stäbe wurde mit einem Maßband gemessen (Fehler jeweils $\pm \SI{0.1}{\cm}$):
\begin{table}[H]
  \centering
  \begin{tabular}{l | c | c | r}
    \# & Länge [cm] \\ \hline
    1 & \SI{29.4}{\cm} \\
    2 & \SI{29.5}{\cm} \\
    3 & \SI{29.8}{\cm} \\
    4 & \SI{29.2}{\cm}
  \end{tabular}
  \caption{Länge der Stäbe}
  \label{tab:stablänge}
\end{table}

Weiterhin standen 5 verschiedene Gewichte (\SI{5}{g}, \SI{10}{g}, \SI{20}{g}, \SI{50}{g}, \SI{100}{g}) zur Verfügung. Die Stäbe wurden an einem Ende horizontal zum Tisch fest eingespannt, wobei das freie Ende sich vor einem Spiegel mit vertikaler Längenskala befand. 

Pro Gewicht wurde nun die Durchbiegung des Stabes gemessen: Zuerst wurde vor jeder Messung der Nullpunkt des Spiegels justiert, sodass er sich mittig hinter dem freien Stabende befand. Dann wurde das Gewicht in eine kleine Schaukel gelegt und diese am freien Ende angebracht. Eventuelle Schwingungen des Stabendes wurden abgewartet und schließlich die vertikale Auslenkung des Stabendes relativ zum Nullpunkt abgelesen. Dabei wurde der Parallaxenfehler gering gehalten, indem beim Ablesen die Spiegelung des Stabes mit dem Stab selbst ausgerichtet wurde.

Der rechteckige Stab wurde sowohl hochkant als auch flachkant eingespannt und gemessen. \\

Die Erwartung ist, dass die Stäbe sich bei höherem angehängten Gewicht stärker biegen. Laut Theorie (\cref{eq:maxbiegung}) sollte der Zusammenhang linear sein (denn $F=mg$). Möglicherweise sind anfangs als schwerer empfundenen Stäbe schlechter zu biegen als die leichteren, weil ein Zusammenhang zwischen innerer Stabilität und Dichte bestehen könnte. Außerdem wird bei flachkantigem Einspannen des 1. Stabes eine stärkere Durchbiegung als bei hochkantigen erwartet.

\begin{table}[H]
  \centering
  \begin{tabular}{l | c | c | c | c | r}
    \# & \SI{5}{g} [mm] & \SI{10}{g} [mm] & \SI{20}{g} [mm] & \SI{50}{g} [mm] & \SI{100}{g} [mm] \\ \hline
    1 flachkant & 1 & 3 & 6 & 14 & 27 \\
    1 hochkant & 0 & 1 & 2 & 3 & 5 \\
    2 & 1 & 2 & 4 & 11 & 22 \\
    3 & 2 & 4 & 7 & 16 & 31 \\
    4 & 1 & 2 & 3 & 5 & 10
  \end{tabular}
  \caption{Messergebnis zur Durchbiegung}
  \label{tab:durchbiegung}
\end{table}
\begin{figure}[H]
  \centering
  \begin{tikzpicture}
    \begin{axis}[
      width=15 cm,
      height=10 cm,
      xmin=0, xmax=100,
      ymin=0, ymax=35,
      xlabel={Gewicht [g]},
      ylabel={Durchbiegung [mm]},
      legend entries={Stab 1 flachkant, Stab 1 hochkant, Stab 2, Stab 3, Stab 4}
    ]
      \addplot[mark=square, only marks, error bars/.cd, y dir=both, y fixed=0.5] table {stab1flach.txt};
      \addplot[mark=*, only marks, error bars/.cd, y dir=both, y fixed=0.5] table {stab1hoch.txt};
      \addplot[mark=x, only marks, error bars/.cd, y dir=both, y fixed=0.5] table {stab2.txt};
      \addplot[mark=+, only marks, error bars/.cd, y dir=both, y fixed=0.5] table {stab3.txt};
      \addplot[mark=otimes, only marks, error bars/.cd, y dir=both, y fixed=0.5] table {stab4.txt};
    \end{axis}
  \end{tikzpicture}
  \caption{Messergebnis zur Durchbiegung}
  \label{fig:durchbiegung}
\end{figure}

Wie erwartet scheint die Durchbiegung linear mit dem angehängten Gewicht zu steigen. Stab 1 hat sich flachkant stärker durchgebogen als hochkant. Der leichteste Stab 3 hat sich stärker als der mittelschwere (2) und dieser wiederum stärker als der schwerste Stab 4 verbogen. \\

Mit \emph{gnuplot} werden nach dem \emph{least-squares}-Verfahren die Werte der einzelnen Stäbe gegen die Funktion $f(x)=a\cdot x$ gefittet. Ausgabe:
\begin{table}[H]
  \centering
  \begin{tabular}{l | c | c}
    \# & a [\si{mm/g}] & Varianz der Residuen \\ \hline
    1 flachkant & \num{0.3} & \num{0.2} \\
    1 hochkant & \num{0.05} & \num{0.3} \\
    2 & \num{0.2} & \num{0.1} \\
    3 & \num{0.3} & \num{.4} \\
    4 & \num{.1} & \num{.5}
  \end{tabular}
  \caption{Linearer Fit der Durchbiegung}
  \label{tab:durchbiegungsfit}
\end{table}
Der von gnuplot berechnete Fehler liegt jeweils außerhalb der signifikanten Stellen.

\cref{eq:maxbiegung} wird umgestellt zu 
\begin{align}
  h_{\text{max}}&=a\cdot m \implies a = \frac{gL^3}{3EI_q} \\
  \implies E&=\frac{mg}{aI_q}\frac{L^3}{3}
  \label{eq:durchbieg_elasti}
\end{align}
und aus der Steigung das Elastizitätsmodul berechnet. Es wird erwartet, dass für Stab 1 flachkant dasselbe Elastizitätsmodul gilt wie für Stab 1 hochkant, da diese eine Materialgröße ist, die unabhängig von der Art der Einspannung sein sollte.
\begin{table}[H]
  \centering
  \begin{tabular}{l | c}
    \# & Elastizitätsmodul [\si{kN/mm^2}] \\ \hline
    1 flachkant & \num{9 \pm 1.5e1} \\
    1 hochkant & \num{8 \pm 0.84e1} \\
    2 & \num{1 \pm 0.5e2} \\
    3 & \num{8 \pm 1.3e1} \\
    4 & \num{2 \pm 1.1e2}
  \end{tabular}
  \caption{Elastizitätsmodul berechnet aus Steigung}
  \label{tab:elastimodul}
\end{table}

Wie erwartet liegt die Differenz zwischen dem Wert für Stab 1 flachkant / hochkant innherhalb des Fehlerbereiches. Die Fehler von Stab 2 und 4 sind relativ gesehen groß, sodass ein Vergleich untereinander sinnlos ist:

\begin{align}
  E_{\text{rel}, 2}&=\frac{\num{.5}}{1}=50\% \\
  E_{\text{rel}, 4}&=\frac{\num{1.1}}{2}=55\%
  \label{eq:elastifehler}
\end{align}

Grund dafür ist, dass es nur eine signifikante Stelle in der Durchbiegungsmessung (\cref{tab:durchbiegung}) gibt.
Sinvoll ausgesagt werden kann, dass der gefühlt leichteste Stab 3 einen deutlich geringeres Elastizitätsmodul hat als der schwerste Stab 4:
\begin{equation}
  \Delta E_{34}=\SI{200}{kN/mm^2} - \SI{80}{kN/mm^2}=\SI{120}{kN/mm^2}
  \label{eq:elastidiff34}
\end{equation}

Laut \footcite{ingenieurwissen} ist das Elastizitätsmodul für verschiedene Werkstoffe:

\begin{table}[H]
  \centering
  \begin{tabular}{l | c}
    Stoff & Elastizitätsmodul [\si{kN/mm^2}] \\ \hline
    Nickellegierungen & 150\ldots 222 \\
    Gusseisen & 66\ldots 172 \\
    Kuper & 100 \ldots 130 \\
    Bronze & 105 \ldots 124 \\
    Messing & 78 \ldots 123 \\
    Aluminiumlegierungen & 68 \ldots 82 \\
    Magnesiumlegierungen & 42 \ldots 47
  \end{tabular}
  \caption{Literaturwerte für Elastizitätsmodul}
  \label{tab:litwertelasti}
\end{table}

Stab 1 liegt im Bereich von Messing, während Stab 3 im Bereich von Aluminiumlegierungen liegt. Beides deckt sich mit dem Aussehen. Bei Stab 2 und 4 ist der Fehler zu hoch, als das die Werte sinnvoll verglichen werden könnten, allerdings legt das sehr ähnliche Aussehen nahe, dass Stab 2 genauso wie Stab 1 aus Messing besteht, während Gewicht und Aussehen von Stab 4 auf Gusseisen hindeuten.
\section{Versuch: Torsionsschwingungen}
\subsection{Schwingung mit Scheibe}
An einem Draht der Länge $L=1,739 \pm 0,002 m$ und dem Durchmesser $d_{Draht}= 0,49 \pm 0,01 mm$ wird eine Scheibe mit dem Radius $R_{Scheibe}= 7,1 \pm 0,05 cm$ und mit der Masse $m_{Scheibe}=2648g$ (angegebener Wert) angehängt und nach einer Auslenkung von $\phi =180$° wird die Dauer für 3 Schwingungsperioden bestimmt.
\begin{table}[H]
  \centering
  \begin{tabular}{l | c | c | c | c}
    Messung & 1 & 2& 3 & 4 \\ \hline
    Schwingungsdauer t & $76,5 \pm 0,5 s $ & $91,03\pm 0,5 s $ & $91,44 \pm 0,5 s $ & $91,06\pm 0,5 s $
  \end{tabular}
  \caption{Schwingungsdauer für drei Perioden mit angehängter Scheibe}
  \label{tab:schwingdauerscheibe}
\end{table}
Im weiteren wird Messung $1$ nicht weiter betrachtet, da ihr Wert deutlich von den anderen Werten abweicht und ein grober Messfehler vorliegt.

So ergibt sich eine durchschnittliche  Schwingungsdauer von $\overline{t}=91,18s$.

Zu Bestimmung des Schubmoduls muss nun die Formel \ref{eq:j_scheibe} in \ref{eq:schubmod} einsetzen und erhält:

\begin{equation}
G_{Scheibe}=\frac{4\pi LmR_{Scheibe}^2}{R^4T^2}= \frac{4\pi\cdot1,739m\cdot2,648kg\cdot(0,071m)^2}{(0,00049m)^4\frac{91,18s}{3}^2}=\num{5,47d9} \frac{N}{m^2}
\end{equation}
Der Messfehler ist
\begin{equation}
\begin{split}
\Delta G_{Scheibe}=[\frac{4\pi mR_{Scheibe}^2}{R^4T^2}\cdot\Delta L)^2+(\frac{4\pi R_{Scheibe}mL}{R^4T^2}\cdot\Delta R_{Scheibe})^2\\+(-\frac{\pi mLR_{Scheibe}^2}{R^5T^2}\cdot\Delta R)^2+(-\frac{2\pi mRL_{Scheibe}^2}{R^4T^3}\cdot\Delta T)^2]^{\frac{1}{2}}\\=\num{9,09d7}\frac{N}{m^2}
\end{split}
\end{equation}
und der relative Fehler ist
\begin{equation}
\frac{\Delta G_{Scheibe}}{G_{Scheibe}}=60,1\%
\end{equation}
\subsection{Schwingung mit Hantel}
Nun wird der Versuchsablauf von ,,Schwingung mit Scheibe'' wiederholt, jedoch wird diesmal anstatt der Scheibe eine Hantel mit verschiebbaren Gewichten an den Seiten an den Draht gehängt.

Die Hantelstange hat eine Masse von $m_{1}=239,05g$ (angegeben), Länge von $l_{1}=25 \pm 0,1 cm$ und einen Radius von $R_{1}=5,99 \pm 0,01mm$.
Die Gewichte sind Zylinder mit einer Masse von $m_{2}=305,2g$ (angegeben), einem Radius von $\overline{R}_{2}=2,5 cm$ und einer Höhe von $\overline{l}_2=19,8mm$.
Durch das reduzieren der angehängten Masse hat sich der Draht nun auf $\overline{L}=1,738 m$ verkürzt.
\begin{table}[H]
  \centering
  \begin{tabular}{l | c | c | c }
    Messung & 1 & 2& 3  \\ \hline
    Schwingungsdauer t & $36,47 \pm 0,5 s $ & $36,59\pm 0,5 s $ & $36,66 \pm 0,5 s $ 
  \end{tabular}
  \caption{Schwingungsdauer für drei Perioden mit angehängter Hantelstange ohne Gewichte}
  \label{tab:schwingdauerstange}
\end{table}
So ergibt sich eine durchschnittliche Periode von $t_{Stange}=12,19s$
\begin{table}[H]
  \centering
  \begin{tabular}{l | c | c | c | c | c}
    Position & 1 & 2& 3 & 4 & 5 \\ \hline
    Abstand zum Aufhängepunkt [cm] & $11,45 \pm 0,1  $ & $9,1 \pm 0,1$ & $6,6 \pm 0,1$ & $4,1 \pm 0,1$ & $1,9 \pm 0,1 $
  \end{tabular}
  \caption{Abstand der einzelnen Massenpositionen zum Aufhängepunkt}
  \label{tab:Abstandgewichte}
\end{table}
\begin{table}[H]
  \centering
  \begin{tabular}{l | c | c | c | c | c}
    Position & 1 & 2& 3 & 4 & 5\\ \hline
    Schwingungsdauer $\overline{t}$ & $104,87s $ & $86,56 s $ & $69,57s $ & $54,30s $ & $43,02s$
  \end{tabular}
  \caption{Schwingungsdauer für drei Perioden mit angehängter Hantel mit Gewichten}
  \label{tab:schwingdauerHantel}
\end{table}

\begin{figure}[H]
  \centering
  \begin{tikzpicture}
    \begin{axis}[
      width=15 cm,
      height=10 cm,
      xmin=0, xmax=80000,
      ymin=0, ymax=1230,
      xlabel={Trägeheitsfaktor [$g\cdot cm^2$]},
      ylabel={Periodenquadrat [$s^2$]},
      legend entries={Stab 1 flachkant}
    ]
      \addplot[mark=square, only marks, error bars/.cd, y dir=both, y fixed=0.5] table {schwingunghantel.txt};
      
    \end{axis}
  \end{tikzpicture}
  \caption{,,Steinerscher Trägheitsfaktor'' $2ma^2$ gegen Periodendauerquadrat aufgetragen}
  \label{fig:durchbiegung}
\end{figure}
Wie aus den Formeln zu erwarten scheint ein linearer Zusammenhang zwischen den Abstandsquadrat ($a^2$) und dem Periodenquadrat zu bestehen. Und es gibt einen Y-Achsenabschnitt der größer als $10s$ ist.

Mit \emph{gnuplot} werden nach dem \emph{least-squares}-Verfahren die Werte gegen die Funktion $f(x)=m\cdot x+b$ gefittet. Ausgabe:

\begin{table}[H]
  \centering
  \begin{tabular}{l | c | c}
    Variabel & Wert \\ \hline
    m & \num{1.292d-2} $\pm 0.0005$ \\
    b & \num{187\pm 6.352} 
  \end{tabular}
  \caption{Linearer Fit}
  \label{tab:durchbiegungsfit}
\end{table}

Wir stellen nun \cref{eq:hanteldauer} um:
\begin{align}
  T^2&=\frac{4\pi^2}{D^*}(J_{\text{Achse}}+2\cdot J_{\text{Scheiben}}+2m_2a^2) \\
  &=m\cdot (2m_2 a^2)+b \\
  \implies D^*&=\frac{4\pi^2}{m} \\
  \implies J_2 &=\frac{1}{2}\left(\frac{D^*}{4 \pi^2}\cdot b - J_1\right)
  \label{eq:hantelergebnis}
\end{align}
und aus der Schwingung ohne Scheiben:
\begin{equation}
  J_1 = \frac{T^2 D^*}{4 \pi^2}
  \label{eq:hantelohnescheibe}
\end{equation}
\begin{table}[H]
  \centering
  \begin{tabular}{l | c | c | c}
    &$D^*$ & $J_1$ & $J_2$ \\ \hline
    aus Fit berechnet & \SI{3.056e-4}{Nm} & \SI{1.15}{gm^2} & \SI{0.149}{gm^2} \\
    aus Abmessungen berechnet & - & \SI{1.247}{gm^2} & \SI{0.0604}{gm^2}
  \end{tabular}
  \caption{Ergebnisse zur Hantelschwingung}
  \label{tab:hantelergebnisse}
\end{table}
